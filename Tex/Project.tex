\documentclass[10pt]{article}
\usepackage{amsmath, amssymb, amsfonts} % برای فرمول‌نویسی
\usepackage{breqn} % برای فرمول‌نویسی
\usepackage{amsthm}% برای اضافه کردن محیط proof
\usepackage{subfigure} % اضافه کردن شکل های زیر شکل ها
\usepackage{setspace} % تعیین فاصله بین خطوط
\usepackage{graphicx} % اضافه کردن عکس
\usepackage{multicol} % حروف چینی چند ستونه
\usepackage[margin=1in]{geometry} % تغییر دادن حاشیه دور
\usepackage{fancyvrb}
\usepackage{tcolorbox}

\graphicspath{{./pics/}}

\usepackage{xepersian}

\settextfont{XB Yas} % برای ست کردن فونت متن
\setdigitfont{XB Zar} % برای ست کردن فونت اعداد

\setcounter{secnumdepth}{4} % Set Sections' Depth
\setcounter{tocdepth}{4}% Set Table Of Content's Depth

\title{پروژه درس سیستم‌های مخابراتی}
\author{نام و نام‌ خانوادگی : امیرحسین یاری
	\\ \\
	شماره دانشجویی : ۹۸۱۰۹۷۱۸ 	\\ \\
	تاریخ : ۲۰ بهمن ۱۴۰۰}
\date{}
\doublespacing

\begin{document}
	
	\maketitle
	\pagebreak
	\tableofcontents
	\newpage
	\section{توضیجات اولیه}
	کد‌های هر بخش در فولدر 
	\lr{Codes}
	قرار دارد و نتایج شبیه سازی در 
	\lr{img}
	قرار دارد.
	همچنین حروف چینی با استفاده از 
	\LaTeX
	بوده و فایل
	\lr{tex}
	مربوط به آن در فولدر 
	\lr{Tex}
	قرار دارد.
	
	\section{پیاده سازی بلوک ها به صورت مجزا}
	توابع مربوطه در فولدر 
	\lr{Codes}
	قرار داد.
	\subsection{\lr{Divide, Combine}}
	تابع 
	\lr{Divide}
	به این شکل نوشته شده که نمونه های فرد و زوج سیگنال را جدا میکند که یکی در میان است و نزدیک ترین حالت به سیستم
	\lr{Real-time}
	است.
	همچنین تابع
	\lr{Combine}
	معکوس ان سیستم است و دو ورودی را یکی در میان با هم ترکیب میکند تا دنباله خروجی را بسازد.
	
	
	\subsection{\lr{PulseShaping}}
	در این بخش تابع 
	\lr{PulseShaping}
	به این شکل نوشته شده است که به ازای یک در دنباله ورودی شکل موج مربوطه به یک را در خروجی میدهد و همینطور برای صفر که در نهایت منجر به خروجی با طول ضرب طول های دو دنباله ورودی و شکل یک میشود. 
	
	
	\subsection{\lr{AnalogMod}}
	این تابع تنها با ضرب ورودی اول در کسینوس و ضرب دومی در سینوس خروجی
	\lr{xc}
	را میدهد.
	
	\subsection{\lr{Channel}}
	این تابع با استفاده از فیلتر 
	\lr{bandpass}
	نوشته شده است که پارامتر
	\lr{steepness}
	تابع که میزان تیز بودن فیلتر است برای بخش سوم قسمت سه
	\lr{0.8}
	گرفته شده اما در مابقی حالات
	\lr{0.98}
	است.
	
	\subsection{\lr{AnalogDemod}}
	این تابع ابتدا ورودی اول را در کسینوس و دومی را در سینوس ضرب میکند و در نهایت از یک فیلتر
	\lr{lowpass}
	عبور میدهد تا به خروجی ها برسد.
	
	
	
	\subsection{\lr{MatchedFilt}}
	در این بخش نیز ابتدا پاسخ ضربه هر شکل را که معکوس شده ان است با ورودی کانولوشن میگیریم و سپس با برسی بزرگ تر بودن کدام نتیجه صفر یا یک بودن بیت خروجی را تخمین میزنیم. 
	
	\newpage
	\subsection{رسم پیام و سیگنال های خواسته شده}
	در این بخش به همان ترتیب که گفته شده سیگنال ها در نمودار زیر اورده شده اند.
	
	
	\begin{figure}[h]
		\centering
		\includegraphics[width=0.9\linewidth]{../pics/1-4}
		\caption{سیگنال های خواسته شده به ترتیب}
		\label{fig:1-4}
	\end{figure}
	
	\newpage
	\section{\lr{USSB Modulation}}
	کد‌ مربوط به این سوال در 
	\lr{Q2.m}
	قرار داد.
	\subsection{نمونه برداری}
	سیگنال های داده شده را با فرکانس
	\lr{$f_s = 10000Hz$}
	نمونه برداری میشوند.
	\subsection{مدوله کردن با دستور
		\lr{ssbmod}}
	سه سیگنال را با فرکانس های داده شده به صورت 
	\lr{USSB}
	مدوله کرده و در نهایت جمع ان ها در سیگنال فرستاده شده
	\lr{sent}
	قرار میگیرد.
	\newpage
	\subsection{رسم سیگنال ها و مدوله شده ان ها}
	در نمودار زیر به ترتیب سه سیگنال و مدوله شده ان ها و در نهایت سیگنال ارسالی رسم شده اند.
	
	\begin{figure}[h]
		\centering
		\includegraphics[width=0.7\linewidth]{../pics/2-3}
		\caption{سیگنال های خواسته شده به ترتیب}
		\label{fig:2-3}
	\end{figure}
	
	\newpage
	\subsection{تبدیل فوریه سیگنال های بخش قبل}
	در نمودار زیر تبدیل فوریه سیگنال ها به ترتیب امده است که با روش
	\lr{fft}
	محاسبه شده و در بازه مشخصی رسم شده اند.
	\begin{figure}[h]
		\centering
		\includegraphics[width=0.8\linewidth]{../pics/2-4}
		\caption{تبدیل فوریه سیگنال های خواسته شده به ترتیب}
		\label{fig:2-4}
	\end{figure}
	
	\newpage
	\subsection{کانال نویزی}
	در این بخش انرژی سیگنال محاسبه شده و انرژی نویز بدست میاید که با تابع
	\lr{wgn()}
	یک نویز سفید با توزیع گوسی با طول خواسته شده و انرژی مشخص ساخته میشود که با سیگنال فرستاده شده جمع میشود و سیگنال دریافتی را میسازد.
	
	حال با برسی تبدیل فوریه بازه فیلتر میان نگذر تایین میشود و با فیلتر 
	\lr{bandpass()}
	سیگنال فیلتر شده و در نمودار زیر رسم شده است.
	
	\begin{figure}[h]
		\centering
		\includegraphics[width=0.8\linewidth]{../pics/2-5}
		\caption{تبدیل فوریه سیگنال دریافتی و سیگنال فیلتر شده}
		\label{fig:2-5}
	\end{figure}
	
	
	\newpage
	\subsection{پیام های دریافتی}
	در این بخش ابتدا هر پیام را فیلتر کرده و سپس با دستور 
	\lr{ssbdemod}
	دمادوله کرده و در نهایت در نمودار زیر به ترتیب نمایش میدهیم.
	همانطور که نیز انتظار داریم پیام ها بسیار به پیام فرستاده نزدیک هستند و مقداری نویز دارند.
	
	\begin{figure}[h]
		\centering
		\includegraphics[width=0.7\linewidth]{../pics/2-6}
		\caption{پیام های ارسالی و دریافت شده}
		\label{fig:2-6}
	\end{figure}
	
	\newpage
	\section{\lr{FM Modulation}}
	کد‌ مربوط به این سوال در 
	\lr{Q3.m}
	قرار داد.
	
	\subsection{نمونه برداری}
	از سیگنال داده شده با فرکانس 
	\lr{$f_s = 10000Hz$}
	نمونه برداری میکنیم و سپس با تابع 
	\lr{fmmod()}
	ان را مدوله میکنیم.
	\subsection{\lr{fmdemod}}
	با دستور 
	\lr{fmdemod()}
	سیگنال را بازیابی میکنیم.
	
	\subsection{مشتق گیری}
	در این بخش با شیفت دادن و کم کردن سیگنال از خودش مشتق پیاده سازی شده و سپس با تابع سوال یک پوش سیگنال محاسبه شده و در نهایت سیگنال پیام بازیابی شده است.
	
	\subsection{\lr{Zero Crossing Detector}}
	در این بخش ابتدا عبور از صفر تشخیص داده شده و سپس با تابع 
	\lr{PalseGenerator}
	پالسی به طول 
	\lr{T}
	تولید میشود که در تابع مشخص شده است و در هر لحظه اندازه خروجی تابع به اندازه تعداد عبور صفر در مدت زمان مشخص است که در نهایت دو برابر فرکانس لحظه‌ای را نتیجه میدهد و در زمان های ابتدایی و انتهایی به دلیل کم بودن تعداد داده ها فرکانس قابل اتکا نیست.
	همچنین از فیلتر پایین گذری با فرکانس حدود پهنای باند سیگنال برای حذف نویز و نرم تر کردن خروجی استفاده کردیم.
	
	\newpage
	\subsection{برسی و رسم سه روش}
	در نمودار زیر نتایج سه روش اورده شده است، که همانطور که انتظار داریم روش اول بسیار دقیق و روش مشتق گیری بسیار نویزی اما قابل اتکا در زمان های کم و در نهایت روش اخر بسیار ساده ولی کمی نویز دار است.
	در نهایت تمامی روش ها سیگنال را بازیابی کردند و در نمودار زیر به ترتیب دیده میشوند.
	
	
	\begin{figure}[h]
		\centering
		\includegraphics[width=0.8\linewidth]{../pics/3-5}
		\caption{پیام های بازیابی شده از سه روش}
		\label{fig:3-5}
	\end{figure}
	
\end{document}